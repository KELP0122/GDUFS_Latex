\documentclass[14pt, a4paper, oneside]{ctexart}
\usepackage{amsmath, amsthm, amssymb, appendix, bm, graphicx, hyperref, mathrsfs,geometry,subfigure}
\usepackage{booktabs}
\usepackage{listings}
\usepackage{hyperref}
\usepackage{xcolor}
\usepackage{float}
\usepackage{cases}
\usepackage{xeCJK}
\usepackage{ruby}
\usepackage{fontspec}
\usepackage{xcolor}
\usepackage{titlesec}
\geometry{left=3.17cm,right=3.17cm,bottom =2.54cm,top=2.54cm}
\lstset{
    basicstyle          =   \sffamily,          % 基本代码风格
    keywordstyle        =   \bfseries,          % 关键字风格
    commentstyle        =   \rmfamily\itshape,  % 注释的风格,斜体
    stringstyle         =   \ttfamily,  % 字符串风格
    flexiblecolumns,                % 别问为什么,加上这个
    numbers             =   left,   % 行号的位置在左边
    showspaces          =   false,  % 是否显示空格,显示了有点乱,所以不现实了
    numberstyle         =   \zihao{-5}\ttfamily,    % 行号的样式,小五号,tt等宽字体
    showstringspaces    =   false,
    captionpos          =   t,      % 这段代码的名字所呈现的位置,t指的是top上面
    frame               =   lrtb,   % 显示边框
}

\lstdefinestyle{Python}{
    language        =   Python, % 语言选Python
    basicstyle      =   \zihao{-5}\ttfamily,
    numberstyle     =   \zihao{-5}\ttfamily,
    keywordstyle    =   \color{blue},
    keywordstyle    =   [2] \color{teal},
    stringstyle     =   \color{magenta},
    commentstyle    =   \color{red}\ttfamily,
    breaklines      =   true,   % 自动换行,建议不要写太长的行
    columns         =   fixed,  % 如果不加这一句,字间距就不固定,很丑,必须加
    basewidth       =   0.5em,
}

\linespread{1.5}
\newtheorem{theorem}{定理}[section]
\newtheorem{definition}[theorem]{定义}
\newtheorem{lemma}[theorem]{引理}
\newtheorem{corollary}[theorem]{推论}
\newtheorem{example}[theorem]{例}
\newtheorem{proposition}[theorem]{命题}
\renewcommand{\abstractname}{\Large\textbf{Abstract}}

\begin{document}
\thispagestyle{empty}

\begin{figure}[t]
    \centering
    \includegraphics[width=13cm]{\logo\logo.png}
\end{figure}

\vspace*{\fill}
    \begin{center}
        \Huge\textbf{This is title}
    \end{center}
\vspace*{\fill}

\begin{table}[b] %填写个人信息
    \centering
    \large
    \begin{tabular}{ll}
    \textbf{Course:} &  \\
    \textbf{Name:} &  \\
    \textbf{Class:} &  \\
    \textbf{Student ID:}&  \\
    \textbf{Date:} &   \\
    \end{tabular}
\end{table}

\newpage

\thispagestyle{empty}
\begin{abstract}
This is abstract.
    
    \par\textbf{Keys:}Key words here;
\end{abstract}

\newpage
\pagenumbering{Roman}
\setcounter{page}{1}
\renewcommand\contentsname{Catalogue}
\tableofcontents

\newpage
\setcounter{page}{1}
\pagenumbering{arabic}

\section{Introduction}
This is introduction

\section{section1}
\subsection{subsection1}

\newpage
\renewcommand\refname{Reference}
\begin{thebibliography}{99}
\bibitem{ref1}This is a reference.
\end{thebibliography}
\LaTeX で日本語を書きましょう!
\end{document}